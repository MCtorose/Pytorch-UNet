% 导言区
\documentclass[utf8 10pt]{article}  % book, report, letter  ctexart ctexbook
\usepackage{ctex} % 中文软件包
\usepackage{graphicx}%图像包
\usepackage{amsmath}
\usepackage{amsfonts}
\newcommand\degree{^\circ}

\newcommand{\myfont}{\textit{\textbf{\textsf{Fancy Text}}}}
\title{\heiti 杂谈勾股定理}
\author{\kaishu 张三}
\date{\today}


% 正文区(文稿区)
\begin{document}
    % 字体族设置
    % 罗马字体
    \textrm{Roman Family}
    \rmfamily Roman Family
    % 无衬线字体
    \textsf{}
    \sffamily
    % 打字机字体
    \texttt{}
    \ttfamily

    % 字体系列设置(粗细,宽度)
    \textmd{Medium Series}
    \textbf{Boldface Series}

    {\mdseries Medium Series}
    {\bfseries Boldface Series}

    % 字体形状(直立,斜体,伪斜体,小型大写)
    \textup{Upright shape}   {\upshape Upright shape}
    \textit{Italic shape}    {\itshape Italic shape}
    \textsl{slanted shape}   {\slshape slanted shape}
    \textsc{small caps shape}{\scshape small caps shape}


    % 中文字体
    {\songti}
    {\heiti}
    {\kaishu}
    {\fangsong}

    %使用\bf{粗体} \it{斜体}代替\textbf{粗体}和\textit{斜体}可以使中文的粗体和斜体显示正常
    中文字体的\textbf{粗体}与\textit{斜体}

    中文字体的\bf{粗体}与\it{斜体}.

    \sffamily
    % 字体大小
    {\tiny 123}\\
    {\scriptsize 123}\\
    {\footnotesize 123}\\
    {\small 123}\\
    {\normalsize 123}\\
    {\larg 123}\\
    {\Large 123}\\
    {\LARGE 123}\\
    {\huge 123}\\
    {\Huge 123}\\

    % 中文字号设置命令
    \zihao{5} 你好??

    \myfont



    \maketitle
    hello world\LaTeX.
    % $数学模式$     $$居中显示$$
    Let $f(x)$ be defined by the formula

    $$f(x)=3x^2+x-1$$ which is a polynomial of degree 2.


    直角三角形:设直角三角形$ABC$,其中$\angle C=90\degree$,则有:
    \begin{equation}
        AB^2=BC^2+AC^2\label{eq:equation}
    \end{equation}



    \tableofcontents


    \chapter{绪论}


    \section{引言}


    \section{实验结果}

    \subsection{实验条件}

    \subsubsection{实验过程}


    \chapter{结束}


    \section{致谢}


    % 特殊符号
    空行分段,多个空行等于1个
    自动缩进,绝对不能使用空格代替
    英文中多个空格处理为1个空格,中文空格将会被忽略
    汉字与其他字符的间距会自动由Xelatex处理
    禁止使用中文全角空格

    ` 左单引号
    ' 右单引号
    `` 左双引号
    '' 右双引号
    -
    --
    ---
    \\ 换行\\
    空\quad 格
    两个\qquad 空格

    \#
    \$
    \%


    % latex引用图片
    \graphicspath{{../result/}}   % 指定图片所在路径(搜索路径)
    \includegraphics[scale=0.3,angle=45]{01}          % 导入图片并配置相关可选配置

    % 表格
    \begin{tabular}{|l |c|| c| c| p{1.5cm}|}     %单竖线,双数线   l左对齐  c居中  r右对齐  p{1.5cm}指定列宽度
        \hline \hline% 双横线
        姓名 & 语文 & 数学 & 英语 & 备注 \\
        \hline % 单横线
        张三 & 10 & 20 & 30 & 优秀 \\
        \hline
    \end{tabular}


    % 浮动体环境
    \LaTeX{}中的插图见(图\ref{fig:fig-01})       %引用
    \begin{figure}[htbp] % 允许各个位置  here此处 top页顶 bottom页底 page独立一页
        \centering
        \includegraphics[scale=0.8]{01}
        \caption{焊缝图片}\label{fig:fig-01}  %设置标签
    \end{figure}


    在\LaTeX{}中的表格成绩单见(表\ref{tab:tab-score})
    \begin{table}[htbp]
        \centering
        \caption{成绩单}
        \begin{tabular}{|l |c|| c| c| p{1.5cm}|}     %单竖线,双数线   l左对齐  c居中  r右对齐  p{1.5cm}指定列宽度
            \hline% 双横线
            姓名 & 语文 & 数学 & 英语 & 备注 \\
            \hline % 单横线
            张三 & 10 & 20 & 30 & 优秀 \\
            \hline
        \end{tabular}\label{tab:tab-score}
    \end{table}


    % 数学公式
    % 数学模式 $$
    $a+b=c+8-d$ \\
    $a_{x^2-2+1}$\\
    \begin{math}
        a+b=c+8-d\\
        3x^{20}=y\\       % 上标
        a_{x^2-2+1}\\     % 下标
        $a_{x^2-2+1}$
    \end{math}


    \section{希腊字母}\label{sec:}
    \alpha
    \beta
    \gamma
    \epsilon
    \pi
    \omega

    \Gamma
    \Delta
    \Theta
    \Pi
    \Omega


    \section{数学函数}
    \log
    \sin
    \cos
    \arcsin
    \arccos
    \ln

    $y=\sin^2 x+\cos^2 x=1$\\
    $y=\sin^{2x}$\\
    $y=\sin^{2x} x$\\
    \sqrt{2}
    \sqrt[4]{16}


    \section{分式}
    $\frac{3}{4}$
    $3/4$
    $\frac{x}{x^2+x+1}$

    行间公式
    \[a+b=c-0\]
    $$a+b=0-9$$
    \begin{displaymath}
        a+b=c-d
    \end{displaymath}


    自动编号公式equation环境
    \begin{equation}
        a+b=c+d\label{eq:equation2}
    \end{equation}

    不编号公式equation*环境,仍然是居中
    \begin{equation*}
    \end{equation*}

    矩阵
    \begin{matrix}
        0 & 1 \\
        2 & 3
    \end{matrix}

    \begin{pmatrix}
        0 & 1 \\
        2 & 3
    \end{pmatrix}

    \begin{bmatrix}
        0 & 1 \\
        2 & 3
    \end{bmatrix}

    \begin{Bmatrix}
        0 & 1 \\
        2 & 3
    \end{Bmatrix}

    \begin{vmatrix}
        0 & 1 \\
        2 & 3
    \end{vmatrix}

    \begin{Vmatrix}
        0 & 1 \\
        2 & 3
    \end{Vmatrix}


    常用省略号
    \dots
    \vdots
    \ddots



    A=\begin{pmatrix}
          a_{11}^2 & \dots  & a_{1n}^2 \\
          \vdots   & \ddots & \vdots   \\
          0        & \dots  & a_{nn}^2 \\
    \end{pmatrix}_{n \times n}

    行内小矩阵 需要手动加上括号
    \begin{math}
        \left(
        \begin{smallmatrix}
            0 & 1 \\
            2 & 3
        \end{smallmatrix}
        \right)
    \end{math}


    多行公式
    \begin{gather*}
    \end{gather*}

    \begin{gather}
        aq+b=c\\
        c+d=e \times 7 \notag \\   %不进行编号
        d-c=0\\
    \end{gather}

    % &指定对齐位置
    \begin{align}
        a-bbbbbbbbbbbbbbbbb&=c\\
        d-cccccccc&=e\\
        a-iii&=q\\
    \end{align}

    \begin{equation}
        \begin{split}
            a+b&=c+d\\
            &=c+2\\
            &=5+2\\
            &=7\\
        \end{split}\label{eq:equation3}
    \end{equation}


    \begin{equation}
        \mathbb{A}\\% 生成花体字符

        D(x)=\begin{cases}
                 1, & \text{如果 } x \in \mathbb{Q}; \\
                 0, & \text{如果 } x \in \mathbb{R}\setminus\mathbb{Q}.
        \end{cases}\label{eq:equation4}
    \end{equation}





\end{document}










